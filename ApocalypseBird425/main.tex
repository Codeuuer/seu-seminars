\documentclass{article}

% Language setting
% Replace `english' with e.g. `spanish' to change the document language
\usepackage[english]{babel}

% Set page size and margins
% Replace `letterpaper' with `a4paper' for UK/EU standard size
\usepackage[letterpaper,top=2cm,bottom=2cm,left=3cm,right=3cm,marginparwidth=1.75cm]{geometry}

% Useful packages
\usepackage{amsmath}
\usepackage{graphicx}
\usepackage[colorlinks=true, allcolors=blue]{hyperref}

\title{Academic Report on Personality Detection}
\author{09022205 Zhou Zihan, 57122414 Zeng Chenyu, 57122311 Zhou Chongyue, 57122411 Cui Shuaifeng
}

\begin{document}
\maketitle

\section{INTRODUCTION AND BACKGROUND OF AUTOMATIC PERSONALITY DETECTION}

1)  Big-Five Model:  It divides  a person’s personality into five  dimensions:   openness,  conscientiousness,  extroversion, agreeableness  and neuroticism. Each  dimension has  a  score indicating  the  degree  of  that  dimension  in  the  individual. The Big-Five model believes that these five dimensions can cover the main aspects of personality and are universal across cultures.  As  the  most  widely  used  model  in  the  study  of personality  traits,  the  Big-Five  model  is  the  core  theory  of personality traits and has far-reaching influence in personal- ity psychology, industrial and organizational psychology and other disciplines

\section{ RECENT ADVANCES IN AUTOMATIC PERSONALITY
DETECTION}

\subsection{Attentive Neural Networks}

1) Personality Recognition on Monologues and Multiparty
Dialogues Using Attentive Networks: Previous studies on
automatic personality recognition have focused on using traditional classification models with linguistic features. However,
the attention neural network with context embedding can play
a greater role in it, but this aspect has not been deeply explored
at present.

\subsection{The Guidance of Psychology}

1) Psychological Questionnaire enhanced Network: Personality is a concept of psychology, and the study
of personality is an important foundation of psychology. The
traditional method of personality testing is to ask the subject
to answer a questionnaire carefully designed by psychologists
to judge the personality traits of people. In recent years, the
research direction of predicting personality traits based on
online posts on social media has gained great interest. Most
research methods employ deep earning models or pre-trained
language models and rely on the models to extract potential
personality clues from the text pieced together from posts,
which is a process completely devoid of any psychological
knowledge. To some extent, this design limits the performance
of the model.

\begin{figure}
\centering
\includegraphics[width=0.25\linewidth]{图片1.png}
\caption{\label{fig:图片}latex.}
\end{figure}

\begin{table}
\centering
\begin{tabular}{l|r}
Item & Quantity \\\hline
Widgets & 42 \\
Gadgets & 13
\end{tabular}
\caption{\label{tab:widgets}An table.}
\end{table}

\subsection{Mathematics}
\[S_n = \frac{X_1 + X_2 + \cdots + X_n}{n}
      = \frac{1}{n}\sum_{i}^{n} X_i\]

\bibliographystyle{alpha}
\bibliography{sample}

\begin{thebibliography}{99}   
    \bibitem{r1}Yang, Tao and Deng, Jinghao and Quan, Xiaojun and Wang, Qifan,
“Orders are unwanted: dynamic deep graph convolutional network
for personality detection,” Proceedings of the AAAI Conference on
Artificial Intelligence, vol. 37, pp. 13896–13904, 2023.
    \bibitem{r2}Lynn, Veronica and Balasubramanian, Niranjan and Schwartz,“ ierarchical modeling for user personality prediction: The role of message-level attention,”Proceedings of the 58th annual meeting of the association for
computational linguistics, pp.5306–5316,2020.
\end{thebibliography}

\end{document}